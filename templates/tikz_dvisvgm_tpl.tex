% << name >>
% Version: << version >>
% Description: << description >>

\documentclass[tikz,dvisvgm]{standalone}

\usepackage[hypertex]{hyperref} % use this for links with dvisvgm
\usetikzlibrary{positioning}
\usetikzlibrary{shapes.geometric}

\usepackage{fontspec}
\newfontfamily\defaultfont[Scale=1.0]{Inter} % use that for custom font
\newfontfamily\monospace[Scale=1.0]{Ubuntu Mono}

\newcommand{\svgnode}[1]{\special{dvisvgm:raw <g class='node' id='#1'>}}
\newcommand{\svgedge}[1]{\special{dvisvgm:raw <g class='edge' id='#1'>}}
\newcommand{\svggend}{\special{dvisvgm:raw </g>}}


\definecolor{<<label.id>>}{HTML}{<<substr(label.color,1,-1)>>}


\begin{document}

\begin{tikzpicture}[
             font=\defaultfont,
             >=latex,
             every edge/.style={draw, line width=3pt,opacity=0.5},
             hw_edge/.style={dashed}, 
             topic_edge/.style={color=blue,text=black}, 
             service_edge/.style={color=orange,text=black}, 
             action_edge/.style={color=purple,text=black}, 
             node/.style={draw, rounded corners, align=center, inner sep=5pt, fill=black!20},
             label/.style={midway, align=center, fill=white,opacity=0.8},
             topic/.style={midway, align=center, font=\monospace, fill=white,opacity=0.8},
             service/.style={midway, align=center, font=\monospace, fill=white,opacity=0.8},
             action/.style={midway, align=center, font=\monospace, fill=white,opacity=0.8}]


    \special{dvisvgm:raw
    <script type="text/javascript">
    function animateEdge(node1, node2) {
    var animation = document.getElementById('animation-' + node1 + '-' + node2);
    if (!animation) {
    var edge = document.getElementById(node1 + '-' + node2);
    var path = edge.getElementsByTagName('path')[0];
    path.id = 'path-' + node1 + '-' + node2;
    var circle = document.createElementNS('http://www.w3.org/2000/svg', 'circle');
    circle.setAttribute('r', '1');
    circle.setAttribute('fill', 'red');
    var animateMotion = document.createElementNS('http://www.w3.org/2000/svg', 'animateMotion');
    animateMotion.setAttribute('dur', '5s');
    animateMotion.setAttribute('id', 'animation-' + node1 + '-' + node2);
    animateMotion.setAttribute('repeatCount', 'indefinite');
    var mpath = document.createElementNS('http://www.w3.org/2000/svg', 'mpath');
    mpath.setAttribute('href', '\#path-' + node1 + '-' + node2);
    animateMotion.appendChild(mpath);
    circle.appendChild(animateMotion);
    edge.parentNode.appendChild(circle);
    }
    }
    function animateEdges() {
    animateEdge('node1', 'node2');
    }
    document.addEventListener('DOMContentLoaded', animateEdges);
    </script>
    }

    \svgnode{<< node.id >>}
    \node at (<< to_mm(node.x) >>mm, -<< to_mm(node.y) >>mm) [node, anchor=north west, minimum width=<< to_mm(node.width) >>mm, minimum height=<< to_mm(node.height) >>mm, fill=<< node.label >>!50] (<< node.id >>) {<< tex_escape(node.name) >>};
    \svggend



    \svgedge{<< edge.id >>}
    \path (<<edge.from>>) edge[->,<< edge.type >>_edge, in=<<edge.in_angle>>, out=<<edge.out_angle>>, looseness=0.4] node[<<edge.type>>]{<<tex_escape(edge.name)>>} (<<edge.to>>);
     \svggend



\end{tikzpicture}
\end{document}
